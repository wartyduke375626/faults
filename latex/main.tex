\documentclass[12pt, twoside]{book}
%\documentclass[12pt, oneside]{book}  % jednostranna tlac

%spravne nastavenie okrajov
\usepackage[a4paper,top=2.5cm,bottom=2.5cm,left=3.5cm,right=2cm]{geometry}
%zapnutie fontov pre UTF8 kodovanie
\usepackage[utf8]{inputenc}
\usepackage[T1]{fontenc}

%zapnutie slovenskeho delenia slov
%a automatickych nadpisov ako Obsah, Obrázok a pod. v slovencine
\usepackage[slovak]{babel} % vypnite pre prace v anglictine!

%nastavenie riadkovania podla smernice
\linespread{1.25} % hodnota 1.25 by mala zodpovedat 1.5 riadkovaniu

% balicek na vkladanie zdrojoveho kodu
\usepackage{listings}
% ukazky kodu su cislovane ako Listing 1,2,...
% tu je Listing zmenene na Algoritmus 1,2,...
\renewcommand{\lstlistingname}{Algoritmus}
% nastavenia balicka listings
% mozete pridat aj language=...
% na nastavenie najcastejsie pouzivaneho prog. jazyka
% takisto sa da zapnut cislovanie riadkov
\lstset{frame=lines}

% balicek na vkladanie obrazkov
\usepackage{graphicx}
% balicek na vkladanie celych pdf dokumentov, tu zadanie
\usepackage{pdfpages}
% balicek na vkladanie diagramov
\usepackage{pgfplots}
\pgfplotsset{width=0.6\textwidth,compat=1.9}
% balicek na spravne formatovanie URL
\usepackage{url}
% balicek na hyperlinky v ramci dokumentu
% zrusime farebne ramiky okolo liniek aby pdf
% vyzeralo rovnako ako tlacena verzia
\usepackage[hidelinks,breaklinks]{hyperref}



% -------------------
% --- Definicia zakladnych pojmov
% --- Vyplnte podla vasho zadania, rok ma byt rok odovzdania
% -------------------
\def\mfrok{2023}
\def\mfnazov{Útoky na hardvér\\pomocou indukovania chýb}
\def\mftyp{Bakalárska práca}
\def\mfautor{Dennis Vita}
\def\mfskolitel{RNDr. Richard Ostertág, PhD. }

%ak mate konzultanta, odkomentujte aj jeho meno na titulnom liste
\def\mfkonzultant{tit. Meno Priezvisko, tit. }  

\def\mfmiesto{Bratislava, \mfrok}

% študenti BIN a DAV odkomentujú príslušnú dvojicu riadkov
\def\mfodbor{ Informatika}
\def\program{ Informatika }
% pre BIN:
%\def\mfodbor{ Informatika a Biológia }
%\def\program{ Bioinformatika }
% pre DAV:
%\def\mfodbor{ Informatika a Matematika } 
%\def\program{ Dátová veda }

% Ak je školiteľ z FMFI, uvádzate katedru školiteľa, zrejme by mala byť aj na zadaní z AIS2
% Ak máte externého školiteľa, uvádzajte Katedru informatiky 
\def\mfpracovisko{ FMFI.KI - Katedra informatiky }

\begin{document}     
\frontmatter
\pagestyle{empty}

% -------------------
% --- Obalka ------
% -------------------

\begin{center}
\sc\large
Univerzita Komenského v Bratislave\\
Fakulta matematiky, fyziky a informatiky

\vfill

{\LARGE\mfnazov}\\
\mftyp
\end{center}

\vfill

{\sc\large 
\noindent \mfrok\\
\mfautor
}

\cleardoublepage
% --- koniec obalky ----

% -------------------
% --- Titulný list
% -------------------


\noindent

\begin{center}
\sc  
\large
Univerzita Komenského v Bratislave\\
Fakulta matematiky, fyziky a informatiky

\vfill

{\LARGE\mfnazov}\\
\mftyp
\end{center}

\vfill

\noindent
\begin{tabular}{ll}
Študijný program: & \program \\
Študijný odbor: & \mfodbor \\
Školiace pracovisko: & \mfpracovisko \\
Školiteľ: & \mfskolitel \\
% Konzultant: & \mfkonzultant \\
\end{tabular}

\vfill


\noindent \mfmiesto\\
\mfautor

\cleardoublepage
% --- Koniec titulnej strany


% -------------------
% --- Zadanie z AIS
% -------------------
% v tlačenej verzii s podpismi zainteresovaných osôb.
% v elektronickej verzii sa zverejňuje zadanie bez podpisov
% v pracach v angličtine anglické aj slovenské zadanie

\newpage 
\setcounter{page}{2}
\includepdf{images/zadanie.pdf}

% --- Koniec zadania


% -------------------
%   Poďakovanie - nepovinné
% -------------------
\newpage 
\pagestyle{plain}
~

\vfill
{\bf Poďakovanie:} Tu môžete poďakovať školiteľovi, prípadne
ďalším osobám, ktoré vám s prácou nejako pomohli, poradili,
poskytli dáta a podobne.

% --- Koniec poďakovania

% -------------------
%   Abstrakt - Slovensky
% -------------------
\newpage 
\section*{Abstrakt}


Slovenský abstrakt v rozsahu 100-500 slov, jeden odstavec. Abstrakt
stručne sumarizuje výsledky práce. Mal by byť pochopiteľný pre bežného
informatika. Nemal by teda využívať skratky, termíny alebo označenie
zavedené v práci, okrem tých, ktoré sú všeobecne známe.

\paragraph*{Kľúčové slová:} jedno, druhé, tretie (prípadne štvrté, piate)
% --- Koniec Abstrakt - Slovensky


% -------------------
% --- Abstrakt - Anglicky 
% -------------------
\newpage 
\section*{Abstract}

Abstract in the English language (translation of the abstract in the
Slovak language).


\paragraph*{Keywords:} 

% --- Koniec Abstrakt - Anglicky

% -------------------
% --- Predhovor - v informatike sa zvacsa nepouziva
% -------------------
%\newpage 
%
%\chapter*{Predhovor}
%
%Predhovor je všeobecná informácia o práci, obsahuje hlavnú charakteristiku práce 
%a okolnosti jej vzniku. Autor zdôvodní výber témy, stručne informuje o cieľoch 
%a význame práce, spomenie domáci a zahraničný kontext, komu je práca určená, 
%použité metódy, stav poznania; autor stručne charakterizuje svoj prístup a svoje 
%hľadisko. 
%
% --- Koniec Predhovor


% -------------------
% --- Obsah
% -------------------

\newpage 

\tableofcontents

% ---  Koniec Obsahu

% -------------------
% --- Zoznamy tabuliek, obrázkov - nepovinne
% -------------------

\newpage 

\listoffigures
\listoftables

% ---  Koniec Zoznamov

\mainmatter
\pagestyle{headings}


\input 00-uvod.tex 

\input 01-teoria.tex

\input 02-hardver.tex

\input 03-utoky.tex

\input 04-CTF.tex

\input 05-zaver.tex

% -------------------
% --- Bibliografia
% -------------------


\newpage	

\backmatter

\thispagestyle{empty}
\clearpage

\bibliographystyle{plain}
\bibliography{literatura} 

%Prípadne môžete napísať literatúru priamo tu
%\begin{thebibliography}{5}
 
%\bibitem{br1} MOLINA H. G. - ULLMAN J. D. - WIDOM J., 2002, Database Systems, Upper Saddle River : Prentice-Hall, 2002, 1119 s., Pearson International edition, 0-13-098043-9

%\bibitem{br2} MOLINA H. G. - ULLMAN J. D. - WIDOM J., 2000 , Databasse System implementation, New Jersey : Prentice-Hall, 2000, 653s., ???

%\bibitem{br3} ULLMAN J. D. - WIDOM J., 1997, A First Course in Database Systems, New Jersey : Prentice-Hall, 1997, 470s., 

%\bibitem{br4} PREFUSE, 2007, The Prefuse visualization toolkit,  [online] Dostupné na internete: <http://prefuse.org/>

%\bibitem{br5} PREFUSE Forum, Sourceforge - Prefuse Forum,  [online] Dostupné na internete: <http://sourceforge.net/projects/prefuse/>

%\end{thebibliography}

%---koniec Referencii

% -------------------
%--- Prilohy---
% -------------------

%Nepovinná časť prílohy obsahuje materiály, ktoré neboli zaradené priamo  do textu. Každá príloha sa začína na novej strane.
%Zoznam príloh je súčasťou obsahu.
%
\input priloha-A.tex

\input priloha-B.tex

\end{document}






