\chapter*{Úvod} % chapter* je necislovana kapitola
\addcontentsline{toc}{chapter}{Úvod} % rucne pridanie do obsahu
\markboth{Úvod}{Úvod} % vyriesenie hlaviciek

Za posledné roky výrazne vzrástlo využitie programom riadených zariadení, či už v~domácnosti, produkčnej sfére alebo v štátnej, či celosvetovej infraštruktúre. Fungovanie každého programu v konečnom dôsledku zabezpečuje hardvér, ktorý na tej najnižšej úrovni jednotlivé inštrukcie programu vykonáva. 

Pri návrhu a implementácii softvéru sa väčšinou predpokladá, že hardvér, na ktorý bude program nasadený bude jednotlivé inštrukcie vykonávať bezchybne a presne tak, ako výrobca daného hardvéru uvádza. Za bežných podmienok je väčšinou tento predpoklad skutočne naplnený. Vplyv externých fyzikálnych faktorov, ktoré sú mimo špecifikácie udávanej výrobcom, ako napríklad prudká a krátka zmena napätia, elektromagnetické rušenie a mnoho ďalších, však môže mať za následok chybné vykonanie aktuálnej inštrukcie a tým spôsobiť nedefinované správanie bežiaceho procesu. Takýmto spôsobom možno cielene vyvolať konkrétnu chybnú operáciu v správnom čase a zaútočiť tak na dané zariadenie, napríklad s účelom obísť bezpečnostný mechanizmus. Takýto typ útokov na hardvér sa nazýva útok indukovaním (vyvolaním) chyby (angl. fault injection attack).

Špeciálna pozornosť pri útokoch pomocou indukovania chýb je venovaná vnoreným (angl. embedded) zariadeniam, ktoré sú okrem iného používané aj na obsluhu bezpečnostných systémov ako napríklad inteligentné zámky, bezpečnostné kamery a pod. Funkčnosť veľkej časti bezpečnostných systémov závisí aj na koncových zariadeniach. Cielené ovplyvnenie ich správania môže predstavovať bezpečnostnú hrozbu. 

Útokmi pomocou indukovania chýb sa zaoberá veľké množstvo článkov a prác \cite{crowbars, clock, lowcost, crypto, emfi, powerGlitch, AntiFI}, ktorých výsledkom sú často úspešné útoky. Náklady na realizáciu útoku pomocou indukovania chýb sa výrazne líšia. Väčšina existujúcich prác využíva pre implementáciu samotného útoku stredné až veľmi vysoké náklady (rádovo od niekoľko stoviek až po desaťtisíce eur). Hlavným cieľom tejto práce je práve overenie či aj veľmi lacný hardvér, ktorého cena sa pohybuje rádovo v desiatkach eur je použiteľný pre útoky pomocou indukovania chýb.

V kapitole \ref{kap:teoria} predstavíme základné techniky indukovania chýb aj stručne vysvetlíme ich základné princípy. V nasledujúcej kapitole \ref{kap:hardver} popíšeme kľúčové aspekty zvoleného hardvéru, ktoré budú podstatné pri implementácií jednotlivých útokoch. Kapitola \ref{kap:utoky} predstavuje hlavnú časť práce, v ktorej sa pokúsime implementovať vybrané útoky. Zároveň sa analýzou pokúsime určiť aký typ chyby (vynechanie inštrukcie, ovplyvnenie výsledku, stavu pamäte a pod.) možno lacným hardvérom dosiahnuť a aká presnosť načasovania je pri útoku potrebná. Na záver, v kapitole \ref{kap:CTF}, aplikujeme tieto znalosti a demonštrujeme úspešný útok na jednoduché programy bežiace na mikrokontroléri ATMega328P \cite{atmegaData}.